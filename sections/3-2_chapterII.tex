\chapter{Content chapter II}
\label{chapterII}

This sample content chapter includes examples of using the template.

\section{Citing}

This is an example of citing your own papers \parencite{paper1,paper2} in addition to just listing them under publications and making them show up in the bibliography. You can reference other papers \parencite{einstein, knuthwebsite,latexcompanion} like this.

Three different citation styles (\texttt{authoryear}, \texttt{numeric}, \texttt{alphabetic}) have been enabled in the template. To select your peferred citation style, uncomment the corresponding \texttt{style=...} line in the "Bibliography" section in \texttt{phdstyle.sty}.

While the template uses \texttt{biblatex}\footnote{\url{https://www.overleaf.com/learn/latex/Bibliography_management_with_bibtex}} to manage bibliography and citations, compatibility mode with \texttt{natbib} has been enabled. Therefore, you can use \texttt{biblatex} commands such as \texttt{\textbackslash parencite} or \texttt{\textbackslash textcite}, or their respective \texttt{natbib} equivalents \texttt{\textbackslash citep} or \texttt{\textbackslash citet}. If you experience any issues with the \texttt{natbib} commands, \texttt{biblatex} commands may be more stable.

In case of \texttt{numeric} and \texttt{alphabetic} styles, \texttt{\textbackslash cite}, \texttt{\textbackslash parencite} (or \texttt{\textbackslash citep}) are equivalent and will just show the citation in square brackets. The three commands differ, however, for the \texttt{authoryear} style:

\begin{itemize}
    \item \texttt{\textbackslash cite} will mention the author and year without using any parenthesis: \cite{einstein}
    \item \texttt{\textbackslash parencite} or \texttt{\textbackslash citep} will put the entire citation into parenthesis: \parencite{einstein}
    \item \texttt{\textbackslash textcite} or \texttt{\textbackslash citet} will mention the author in text and put the year into parenthesis (or the number or alphabetic reference into square brackets when using other citation styles): \textcite{einstein}
\end{itemize}

\section{Using the glossary}
If you wish to include a glossary, you can define items in \texttt{glossary.sty}.

You can use the defined acronyms with the \texttt{\textbackslash gls} command like this: \gls{ut}, \gls{ny}, \gls{la}, \gls{un}. This will print the acronym and its definition the first time and just the acronym all following times like this: \gls{ut}. To manually set whether only the acronym, only the definiton or both should be printed, commands \texttt{\textbackslash acrshort}, \texttt{\textbackslash acrlong} and \texttt{\textbackslash acrfull} can be used.

The nomenclature items can also be used with the \texttt{\textbackslash gls} command like this: \gls{angelsperarea}, \gls{numofangels}, \gls{areaofneedle}. For more information about using the glossary, check out the documentation\footnote{\url{https://www.overleaf.com/learn/latex/Glossaries}}.

\section{Inserting figures and tables}

% generated by https://www.lipsum.com/
Lorem ipsum dolor sit amet, consectetur adipiscing elit. Aenean nec diam turpis. Integer aliquam purus non mauris faucibus, in luctus ex tristique. Pellentesque consectetur metus neque, et malesuada diam gravida ac. Nullam eu facilisis metus, eget blandit justo. Praesent magna lorem, semper at ornare in, mattis lobortis nulla. Curabitur luctus faucibus diam, eu molestie tortor facilisis vitae. Nulla tempus iaculis quam nec dignissim. Mauris tincidunt gravida Fig.~\ref{fig:my_label} felis.

\begin{figure}[htp]
    \centering
    \includegraphics[width=7cm]{assets/figures/overleaf_wide_colour_green_bg.png}
    \caption{A sample figure}
    \label{fig:my_label}
\end{figure}

Vestibulum est turpis, blandit vitae aliquet id, hendrerit eget est. Class aptent taciti sociosqu ad litora torquent per conubia nostra, per inceptos himenaeos. Duis ut ultricies elit. Pellentesque vitae blandit urna. Fusce sollicitudin, leo sed egestas ornare, eros ipsum suscipit lacus, sed faucibus turpis nisi vitae felis. Curabitur vel hendrerit mauris. Cras blandit purus elit, sit amet convallis eros Table~\ref{tab:my_label} scelerisque.

\input{assets/tables/sample_table}

Nam feugiat, nisl vel consectetur tempus, metus eros gravida orci, a semper odio odio sed erat. Donec in metus orci. Proin ultrices condimentum ligula, et posuere tortor ultricies vel. Sed dictum id tellus in consectetur. Donec scelerisque ullamcorper eros at hendrerit. Vivamus varius laoreet felis, non congue enim blandit ac. Vivamus maximus dolor eget mauris faucibus, a tempus mi facilisis. Integer mollis blandit dolor nec varius.