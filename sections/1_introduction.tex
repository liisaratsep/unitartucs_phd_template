\chapter{Introduction}

This template is designed to format doctoral theses by PhD students at the Institute of Computer Science. It contains the main structure and functionality that is required for writing the thesis.

The University of Tartu Press general guidelines\footnote{\url{https://tyk.ee/et/nouded-kasikirjadele}, December 3, 2024} have been followed in the template. Additional requirements for the thesis can be found in the Regulations for Doctoral Studies and Study Regulations\footnote{\url{https://sisu.ut.ee/ope/requirements-doctoral-thesis/?lang=en}}.

The template is split into multiple files and folders:

\begin{itemize}
\item \texttt{phdunitartu.cls}: A class file for both monograph and collection of publications type of theses;
\item \texttt{phdstyle.sty}: A style file for both monograph and collection of publications type of theses;
\item \texttt{main.tex}: The main file for the template with the thesis details need to be filled here;
\item \texttt{assets/glossary.sty}: a file containing all glossary entries.
\item \texttt{bibliographies/bibliography.bib}: A bibliography file for cited works
\item \texttt{bibliographies/publications.bib}: A bibliography file for author's publications
\item \texttt{sections/}: directory to keep the main content of thesis in, children files reflect the main document structure
\item \texttt{extras/}: other chapters required in the thesis
\item \texttt{assets/figures/}: to store the figures for the document
\item \texttt{assets/publications/}: a directory to store the PDF files of the publications to be included
\item \texttt{assets/tables/}: a directory to store the tables
\item \texttt{README.md}: a readme file documenting how the template is maintained.
\end{itemize}

The template contains two sample content chapters. Chapter \ref{chapterI} contains information about permitted changes to the template and a list of things that the author is not allowed to change. Chapter \ref{chapterII} provides examples of using the glossary, references, including figures and tables, and other commands in the template.