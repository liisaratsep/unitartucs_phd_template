\chapter{Content chapter I}
\label{chapterI}

This sample content chapter includes information about permitted changes and a list of things the author is not allowed to change. The information below is not exhaustive, therefore, please familiarize yourself with the UT doctoral thesis requirements\footnote{\url{https://sisu.ut.ee/ope/requirements-doctoral-thesis/?lang=en}}. In case these requirements conflict with the information in this template, the UT requirements should be followed.

\section{Things that should not be changed}

You should \textbf{not} change:

\begin{itemize}
    \item The text area size, as well as horizontal and vertical margins;
    \item The main font size and font itself;
    \item Font sizes and fonts of the titles.
\end{itemize}

Most of the sections (introduction, bibliography, conclusion, summary in Estonian) included are mandatory according to the UT regulations and may not be removed.

Sections \textit{Curriculum Vitae}, \textit{Elulookirjeluds} (CV in Estonian) and \textit{List of original publications} belong to the format of \textit{Dissertationes informaticae Universitatis Tartuensis} series and should not be omitted.

In a dissertation of the collection type, the section \textit{Publications included in the thesis} should list all papers whose reprints are included in the thesis and no others. Other original publications may be listed in separate sections. In case of listing publications with several authors, the author's contribution has to be described in the section \textit{Author’s contribution to the publications}.

Every reprinted publication in section \textit{Publications} should be preceded by a separate page containing its full publication record.

\section{Tolerable changes}

UT Press prefers to get the dissertation PDF files with a4 paper size. This means that the text area is much smaller than the page. The page will be cut to the right size by the UT Press. During writing, however, you may obtain a more adequate appearance of the result if you replace \texttt{a4paper} with \texttt{b5paper} in the class file. This change does not harm any other measures.

The order of sections in the template follows a standard thesis structure and should not be changed in most cases. Please modify it with responsibility and care. Note that the sections and their order are slightly different in monograph and collection types of the thesis. This is intentional. Please make sure to use the right template by calling the \texttt{ThesisType} command with the right value when you start writing.

Numbering of chapters is also intended to remain unchanged. So, an abstract, list of contents, list of figures, tables and abbreviations, bibliography, acknowledgement, summary in Estonian, publications, and CV should not be numbered, while the introduction and conclusion should preferably be numbered. The principle behind this choice is that both the introduction and conclusion are parts of the thesis. If you find this principle not being true, you may omit numbers of the introduction and conclusion, too. The recommended style of numbering uses arabic numbers in the main part and Latin alphabet for appendices. You are free to change it if you prefer.

Y\textbf{ou are expected to change} the values of “Dissertation title”, content chapter and (sub)section titles, appendix titles, “Töö pealkiri” (the thesis title in Estonian in the Estonian summary chapter). Please do not substitute other chapter names, for instance, "Sisukokkuvõte" (Summary in Estonian). Substituting, “Introduction” and “Conclusion” by something more precise is tolerable under good reasons.

Abstract, list of figures, list of tables, list of abbreviations, preface and acknowledgement may be omitted. In a dissertation of the collection type, not all subsections of “List of original oublications” given in the example are mandatory. If you only list publications whose reprints are included in the thesis, please remove the subsection and rename the section title to “Publications included in the thesis”.

The student may use any reference style out of the three standard ones. Likewise, choice of the format of the bibliographic records in the bibliography section, list of publications, and before each reprinted publication (in a thesis of the collection type) is up to the author of the thesis. The same style must be used throughout the thesis.

In a thesis of the collection type, it is recommended to include each publication also in the general list of contents of the thesis, as shown in the example. Alternately, a separate list of contents for the reprinted publications may be created.